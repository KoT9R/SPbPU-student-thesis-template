%%%% Шаблон ВКР <<SPbPU-student-thesis-template>>  %%%%
%%
%%   Создан на основе глубокой переработки шаблона российских кандидатских и докторских диссертаций [1]. 
%%   
%%   Полный список различий может быть получен командами git.
%%   Лист авторов-составителей расположен в README.md файле.
%%   Подробные инструкции по использованию в [1,2].
%%   
%%   Рекомендуем установить TeX Live + TeXstudio
%%   <<Стандартная>> компиляция 2-3 РАЗА с помощью pdflatex + biber (для библиографии)     
%%  
%%%% Student thesis template <<SPbPU-student-thesis-template>> %%%%
%%
%%   Created on the basis of deepl modifification of the Russian candidate and doctorate thesis template [1]. 
%%   
%%   Full list of differences can be achieved by git commands.
%%   List of template authors can be seen in the README.md file.
%%   Detailed instructions of usage, see, please in [1,2].
%%     
%%   [1] github.com/AndreyAkinshin/Russian-Phd-LaTeX-Dissertation-Template 
%%   [2] Author_guide_SPBPU-student-thesis-template.pdf
%%   
%%   It is recommended to install TeX Live + TeXstudio   
%%   Default compilation 2-3 TIMES with pdflatex + biber (for the bibliography)
%%  
\input{template_settings/ch_preamble} % лучше не редактировать / please, keep unmodified

\setcounter{docType}{1} % лучше не редактировать / please, keep unmodified

%%%% Настройки автора / Author settings
%% 
\input{my_folder/my_settings} % добавляем свои команды / update your commands

\begin{document} % начало документа


%%% Внесите свои данные - Input your data
%%
%%
\newcommand{\Author}{Д.М.\,Попеску} % И.О. Фамилия автора 
\newcommand{\AuthorFull}{Попеску Денис Михайлович} % Фамилия Имя Отчество автора
\newcommand{\AuthorFullDat}{Попеску Денису Михайловичу} % Фамилия Имя Отчество автора в дательном падеже (Кому? Студенту...)
\newcommand{\AuthorFullVin}{Попеску Дениса Михайловича} % в винительном падеже (Кого? что?  Програмиста ...)
\newcommand{\AuthorPhone}{+7-912-412-53-15} % номер телефорна автора для оперативной связи  
\newcommand{\Supervisor}{С.Ю.\,Беляев} % И. О. Фамилия научного руководителя
\newcommand{\SupervisorFull}{Беляев Сергей Юрьевич} % Фамилия Имя Отчество научного руководителя
\newcommand{\SupervisorVin}{С.Ю.\,Беляева} % И. О. Фамилия научного руководителя  в винительном падеже (Кого? что? Руководителя ...)
\newcommand{\SupervisorJob}{Доцент} %
\newcommand{\SupervisorJobVin}{Доцента} % в винительном падеже (Кого? что?  Програмиста ...)
\newcommand{\SupervisorDegree}{Кандидат физико-математических наук} %
\newcommand{\SupervisorTitle}{старший научный сотрудник} % 
%%
%%
%Руководитель, утверждающий задание
\newcommand{\Head}{Л.В.\,Уткин} % И. О. Фамилия руководителя подразделения (руководителя ОП)
\newcommand{\HeadDegree}{Директор}% Только должность:   
%Руководитель %ОП 
%Заведующий % кафедрой
%Директор % Высшей школы
%Зам. директора
\newcommand{\HeadDep}{ВШПМиВФ} % заменить на краткую аббревиатуру подразделения или оставить пустым, если утверждает руководитель ОП

%%% Руководитель, принимающий заявление
\newcommand{\HeadAp}{И.О.\,Фамилия} % И. О. Фамилия руководителя подразделения (руководителя ОП)
\newcommand{\HeadApDegree}{Должность руководителя}% Только должность:   
%Руководитель ОП 
%Заведующий кафедрой
%Директор Высшей школы
\newcommand{\HeadApDep}{O} % заменить на краткую аббревиатуру подразделения или оставить пустым, если утверждает руководитель ОП
%%% Консультант по нормоконтролю
\newcommand{\ConsultantNorm}{Л.А.\,Арефьева} % И. О. Фамилия консультанта по нормоконтролю. ТОЛЬКО из числа ППС!
\newcommand{\ConsultantNormDegree}{должность, степень} %   
%%% Первый консультант
\newcommand{\ConsultantExtraFull}{Горовой Владимир Андреевич} % Фамилия Имя Отчетство дополнительного консультанта 
\newcommand{\ConsultantExtra}{В.А.\,Горовой} % И. О. Фамилия дополнительного консультанта 
\newcommand{\ConsultantExtraDegree}{Менеджер продукта} % 
\newcommand{\ConsultantExtraVin}{И.О.\,Фамилию} % И. О. Фамилия дополнительного консультанта в винительном падеже (Кого? что? Руководителя ...)
\newcommand{\ConsultantExtraDegreeVin}{должность, степень} %  в винительном падеже (Кого? что? Руководителя ...)
%%% Второй консультант
\newcommand{\ConsultantExtraTwoFull}{Фамилия Имя Отчетство} % Фамилия Имя Отчетство дополнительного консультанта 
\newcommand{\ConsultantExtraTwo}{И.О.\,Фамилия} % И. О. Фамилия дополнительного консультанта 
\newcommand{\ConsultantExtraTwoDegree}{должность, степень} % 
\newcommand{\ConsultantExtraTwoVin}{И.О.\,Фамилию} % И. О. Фамилия дополнительного консультанта в винительном падеже (Кого? что? Руководителя ...)
\newcommand{\ConsultantExtraTwoDegreeVin}{должность, степень} %  в винительном падеже (Кого? что? Руководителя ...)
\newcommand{\Reviewer}{И.О.\,Фамилия} % И. О. Фамилия резензента. Обязателен только для магистров.
\newcommand{\ReviewerDegree}{должность, степень} % 
%%
%%
\renewcommand{\thesisTitle}{Векторное представление структурированных объектов}
\newcommand{\thesisDegree}{работа бакалавра}% дипломный проект, дипломная работа, магистерская диссертация %c 2020
\newcommand{\thesisTitleEn}{Objects embeddings} %2020
\newcommand{\thesisDeadline}{дд.мм.202X}
\newcommand{\thesisStartDate}{дд.мм.202X}
\newcommand{\thesisYear}{2021}
%%
%%
\newcommand{\group}{3630102/70301} % заменить вместо N номер группы
\newcommand{\thesisSpecialtyCode}{01.03.02}% код направления подготовки
\newcommand{\thesisSpecialtyTitle}{Прикладная математика и информатика} % наименование направления/специальности
\newcommand{\thesisOPPostfix}{03} % последние цифры кода образовательной программы (после <<_>>)
\newcommand{\thesisOPTitle}{Математическое и информационное обеспечение экономической деятельностью}% наименование образовательной программы
%%
%%
\newcommand{\institute}{
Высшая школа прикладной математики и вычислительной физики
%Институт компьютерных наук и~технологий
%Гуманитарный институт
%Инженерно-строительный институт
%Институт биомедицинских систем и технологий
%Институт металлургии, машиностроения и транспорта
%Институт передовых производственных технологий
%Институт прикладной математики и механики
%Институт физики, нанотехнологий и телекоммуникаций
%Институт физической культуры, спорта и туризма
%Институт энергетики и транспортных систем
%Институт промышленного менеджмента, экономики и торговли
}%
%%
%%




%%% Задание ключевых слов и аннотации
%%
%%
%% Ключевых слов от 3 до 5 слов или словосочетаний в именительном падеже именительном падеже множественного числа (или в единственном числе, если нет другой формы) по правилам русского языка!!!
%%
%%
\newcommand{\keywordsRu}{векторное представление, рекомендация, метрика близости, кластеризация} % ВВЕДИТЕ ключевые слова по-русски
%%
%%
\newcommand{\keywordsEn}{vector representation, recommendation, proximity metric, clustering} % ВВЕДИТЕ ключевые слова по-английски
%%
%%
%% Реферат ОТ 1000 ДО 1500 знаков на русский или английский текст
%%
%Реферат должен содержать:
%- предмет, тему, цель ВКР;
%- метод или методологию проведения ВКР:
%- результаты ВКР:
%- область применения результатов ВКР;
%- выводы.

\newcommand{\abstractRu}{В данной работе изложена сущность подхода к получению векторного представления структурированных объектов. Это достигается путем применений моделей из сферы обработки естесственного языка. Приведен пример с предообработкой и анализом реальных данных. Описаны различные модели получения векторного представления слов. Разработана программаня реализация для предообработки данных, обучение модели и составление рекомендаций. Проведен анализ полученного векторного представления, а также кластеризация с целью интерпретации полученных результатов.} % ВВЕДИТЕ текст аннотации по-русски
%%
%%
\newcommand{\abstractEn}{This paper outlines the essence of the approach to obtaining a representation of structured objects. This is achieved by applying models from the natural language processing realm. An example with preprocessing and analysis of real data is given. Various models of using word representation are described. A software implementation has been developed for data preprocessing, training the model and making recommendations. The analysis of the obtained representation, as well as clustering in order to interpret the results.} % ВВЕДИТЕ текст аннотации по-английски


%%% РАЗДЕЛ ДЛЯ ОФОРМЛЕНИЯ ПРАКТИКИ
%Место прохождения практики
\newcommand{\PracticeType}{Отчет о прохождении % 
	%стационарной производственной (технологической (проектно-технологической)) %
	такой-то % тип и вид ЗАМЕНИТЬ
	практики}

\newcommand{\Workplace}{СПбПУ, ИКНТ, ВШИСиСТ} % TODO Rename this variable

% Даты начала/окончания
\newcommand{\PracticeStartDate}{%
дд.мм.гггг%
%	22.06.2020
}%
\newcommand{\PracticeEndDate}{%
	дд.мм.гггг%
%	18.07.2020%
}%
%%

\newcommand{\School}{
	Название высшей школы
%	Высшая школа интеллектуальных систем и~суперкомпьютерных~технологий 
}
\newcommand{\practiceTitle}{Тема практики}


%% ВНИМАНИЕ! Необходимо либо заменить текст аннотации (ключевых слов) на русском и английском, либо удалить там весь текст, иначе в свойства pdf-отчета по практике пойдет шаблонный текст.

%%% Не меняем дальнейшую часть - Do not modify the rest part
%%
%%
%%
%%
\ifnumequal{\value{docType}}{1}{% Если ВКР, то...
	\newcommand{\DocType}{Выпускная квалификационная работа}
	\newcommand{\pdfDocType}{\DocType~(\thesisDegree)} %задаём метаданные pdf файла
	\newcommand{\pdfTitle}{\thesisTitle}
}{% Иначе 
	\newcommand{\DocType}{\PracticeType}
	\newcommand{\pdfDocType}{\DocType} %задаём метаданные pdf файла
	\newcommand{\pdfTitle}{\practiceTitle}
}%
\newcommand{\HeadTitle}{\HeadDegree~\HeadDep}
\newcommand{\HeadApTitle}{\HeadApDegree~\HeadApDep}
\newcommand{\thesisOPCode}{\thesisSpecialtyCode\_\thesisOPPostfix}% код образовательной программы
\newcommand{\thesisSpecialtyCodeAndTitle}{\thesisSpecialtyCode~\thesisSpecialtyTitle}% Код и наименование направления/специальности
\newcommand{\thesisOPCodeAndTitle}{\thesisOPCode~\thesisOPTitle} % код и наименование образовательной программы
%%
%%
\hypersetup{%часть болка hypesetup в style
		pdftitle={\pdfTitle},    % Заголовок pdf-файла
		pdfauthor={\AuthorFull},    % Автор
		pdfsubject={\pdfDocType. Шифр и наименование направления подготовки: \thesisSpecialtyCodeAndTitle. \abstractRu},      % Тема
		pdfcreator={LaTeX, SPbPU-student-thesis-template},     % Приложение-создатель
%		pdfproducer={},  % Производитель, Производитель PDF % будет выставлена автоматически
		pdfkeywords={\keywordsRu}
}
%%
%%
%% вспомогательные команды
\newcommand{\firef}[1]{рис.\ref{#1}} %figure reference
\newcommand{\taref}[1]{табл.\ref{#1}}	%table reference
%%
%%
%% Архивный вариант задания ключевых слов, аннотации и благодарностей 
% Too hard to export data from the environment to pdf-info
% https://tex.stackexchange.com/questions/184503/collecting-contents-of-environment-and-store-them-for-later-retrieval
%заменить NewEnviron на newenvironment для распознавания команды в TexStudio
%\NewEnviron{keywordsRu}{\noindent\MakeUppercase{\BODY}}
%\NewEnviron{keywordsEn}{\noindent\MakeUppercase{\BODY}}
%\newenvironment{abstractRu}{}{}
%\newenvironment{abstractEn}{}{}
%\newenvironment{acknowledgementsRu}{\par{\normalfont \acknowledgements.}}{}
%\newenvironment{acknowledgementsEn}{\par{\normalfont \acknowledgementsENG.}}{}


%%% Переопределение именований %%% Не меняем - Do not modify
%\newcommand{\Ministry}{Минобрнауки России} 
\newcommand{\Ministry}{Министерство науки и высшего образования Российской~Федерации} %с 2020
\newcommand{\SPbPU}{Санкт-Петербургский политехнический университет Петра~Великого}
\newcommand{\SPbPUOfficialPrefix}{Федеральное государственное автономное образовательное учреждение высшего образования}
\newcommand{\SPbPUOfficialShort}{ФГАОУ~ВО~<<СПбПУ>>}
%% Пробел между И. О. не допускается.
\renewcommand{\alsoname}{см. также}
\renewcommand{\seename}{см.}
\renewcommand{\headtoname}{вх.}
\renewcommand{\ccname}{исх.}
\renewcommand{\enclname}{вкл.}
\renewcommand{\pagename}{Pages}
\renewcommand{\partname}{Часть}
\renewcommand{\abstractname}{\textbf{Аннотация}}
\newcommand{\abstractnameENG}{\textbf{Annotation}}
\newcommand{\keywords}{\textbf{Ключевые слова}}
\newcommand{\keywordsENG}{\textbf{Keywords}}
\newcommand{\acknowledgements}{\textbf{Благодарности}}
\newcommand{\acknowledgementsENG}{\textbf{Acknowledgements}}
\renewcommand{\contentsname}{Content} % 
%\renewcommand{\contentsname}{Содержание} % (ГОСТ Р 7.0.11-2011, 4)
%\renewcommand{\contentsname}{Оглавление} % (ГОСТ Р 7.0.11-2011, 4)
\renewcommand{\figurename}{Рис.} % Стиль СПбПУ
%\renewcommand{\figurename}{Рисунок} % (ГОСТ Р 7.0.11-2011, 5.3.9)
\renewcommand{\tablename}{Таблица} % (ГОСТ Р 7.0.11-2011, 5.3.10)
%\renewcommand{\indexname}{Предметный указатель}
\renewcommand{\listfigurename}{Список рисунков}
\renewcommand{\listtablename}{Список таблиц}
\renewcommand{\refname}{\fullbibtitle}
\renewcommand{\bibname}{\fullbibtitle}

\newcommand{\chapterEnTitle}{Сhapter title} % <- input the English title here (only once!) 
\newcommand{\chapterRuTitle}{Название главы}          % <- введите 
\newcommand{\sectionEnTitle}{Section title} %<- input subparagraph title in english
\newcommand{\sectionRuTitle}{Название подраздела} % <- введите название подраздела по-русски
\newcommand{\subsectionEnTitle}{Subsection title} % - input subsection title in english
\newcommand{\subsectionRuTitle}{Название параграфа} % <- введите название параграфа по-русски
\newcommand{\subsubsectionEnTitle}{Subsubsection title} % <- input subparagraph title in english
\newcommand{\subsubsectionRuTitle}{Название подпараграфа} % <- введите название подпараграфа по-русски % Заполнить сведения, 
										 % в т.ч. ключевые слова и аннотацию.

%%% Титульник ВКР / Thesis title 
%%
%% добавить лист в pdf-навигацию 
%% add to pdf navigation menu
%%
\pdfbookmark[-1]{\pdfTitle}{tit}
%%
\thispagestyle{empty}%
\makeatletter
\newgeometry{top=2cm,bottom=2cm,left=3cm,right=1cm,headsep=0cm,footskip=0cm}
\savegeometry{NoFoot}%
\makeatother


% %%% Распечатать версию документа / Print document version
% %%
% \begin{flushright}
% %	\vspace{0pt plus0.1fill}
% 	\boxed{\small
% 		\begin{tabular}{r} 
% 			\textbf{Пример ВКР <<SPbPU-student-thesis-template>>.} %\\ % перенос на новую строку
% 			\textbf{Версия от \today % \; время:  \currenttime. % время версии
% 			}
% 		\end{tabular}
% 	} %end boxed
% %	\vspace*{-5pt} % раскомментировать, если не хватает места
% 	\vspace{0pt plus0.1fill} % раскоментировать, если хватает места
% \end{flushright}

{\centering%
	\Ministry\\
	\SPbPU\\
	{%\bfseries %2020 - указание на изменения, которые могут быть введены в 2020 году
		\institute}
\par}%


\vspace{0pt plus1fill} %число перед fill = кратность относительно некоторого расстояния fill, кусками которого заполнены пустые места


\noindent
\begin{minipage}{\linewidth}
	\vspace{\mfloatsep} % интервал 
	\begin{tabularx}{\linewidth}{Xl}
	&Работа допущена к защите     \\
	&\HeadTitle     \\			
	&\underline{\hspace*{0.1\textheight}} \Head     \\
	&<<\underline{\hspace*{0.05\textheight}}>> \underline{\hspace*{0.1\textheight}} \thesisYear~г.  \\ 
	\end{tabularx}
	\vspace{\mfloatsep} % интервал 	
\end{minipage}


\vspace{0pt plus2fill} %


{\centering%
	
	\MakeUppercase{\bfseries{}\DocType} \\ 
	\MakeUppercase{\thesisDegree}%


%\intervalS% %ОБЯЗАТЕЛЬНО ДОБАВИТЬ ОТСТУП, ЕСЛИ ХВАТАЕТ МЕСТА
{\centering%
	\MakeUppercase{\bfseries{\thesisTitle}}}%

}\par%

%\intervalS% %ОБЯЗАТЕЛЬНО ДОБАВИТЬ ОТСТУП, ЕСЛИ ХВАТАЕТ МЕСТА
%по специальности % для специалистов
\noindent	по направлению подготовки \thesisSpecialtyCodeAndTitle{}\\ \\% для бакалавров и магистров 
%\noindent Направленность  % для специалистов
\noindent	Направленность (профиль)	\thesisOPCodeAndTitle % для бакалавров и магистров
% Лучше по~профилю, но что делать, так составили Положение
\par%





\vspace{4mm plus2fill}%

\noindent
\begin{tabularx}{\linewidth}{lXl}
	Выполнил              &	   &             \\
	студент гр.~\group     &    & \Author     \\[\mfloatsep]

	Руководитель 		  &    &             \\
	\SupervisorJob,		  &    &             \\
	\SupervisorDegree, \\
	\SupervisorTitle 	  &    & \Supervisor \\[\mfloatsep]
	
	Консультант		  &    & 			 \\
	\ConsultantExtraDegree 	  &    & \ConsultantExtra\\[\mfloatsep]
	
	Консультант  &    &  \\   	
	по нормоконтролю  		 	  &    & \ConsultantNorm  % обязателен
\end{tabularx} %


%
\vspace{0pt plus4fill}% 


\begin{center}%
Санкт-Петербург\\
\thesisYear
\end{center}%
\restoregeometry
\newpage					 % Титульный лист
										 % Убираем footnotes, консультанта, если нет

% \input{my_folder/task}					 % Задание 
										 % Для сдачи в высшую школу компилируем двухсторонний My_task.tex 
								 		 % После подписания задания изменение его содержания и оформления запрещено

%% Не менять - Do not modify
%%\input{my_folder/summary_settings} 
\chapter*[Count-me]{Реферат} % * - не нумеруем
\thispagestyle{empty}% удаляем параметры страницы
%\setcounter{sumPageFirst}{\value{page}}
%sumPageFirst \arabic{sumPageFirst}
%
%
%% Возможность проверить другие значения счетчиков - debugging
%\ref*{TotPages}~с.,
%\formbytotal{mytotalfigures}{рисун}{ок}{ка}{ков},
%\formbytotal{mytotaltables}{таблиц}{у}{ы}{},
%There are \TotalValue{mytotalfigures} figures in this document
%There are \TotalValue{mytotalfiguresInApp} figuresINAPP in this document
%There are \TotalValue{mytotaltables} tables in this document
%There are \TotalValue{mytotaltablesInApp} figuresINAPP in this document
%There are \TotalValue{myappendices} appendix chapters in this document
%\total{citenum}~библ. наименований.



%% Для того, чтобы значения счетчиков корректно отобразились, необходимо скомпилировать файл 2-3 раза
На \total{mypages}~c.,  
\formbytotal{myfigures}{рисун}{ок}{ка}{ков},
% \formbytotal{mytables}{таблиц}{у}{ы}{},
% \formbytotal{myappendices}{приложен}{ие}{ия}{ий}%.  

%\noindent
{\MakeUppercase{Ключевые слова: \keywordsRu}.} % Ключевые слова из renames.tex

Тема выпускной квалификационной работы: <<\thesisTitle>>


\abstractRu % Аннотация из renames.tex



\printTheAbstract % не удалять


\total{mypages}~pages, 
\total{myfigures}~figures, 
% \total{mytables}~tables,
% \total{myappendices}~appendices%.

%\noindent
{\MakeUppercase{Keywords: \keywordsEn}.} % Ключевые слова из renames.tex 
	
The subject of the graduate qualification work is <<\thesisTitleEn>>.
	
	
\abstractEn % Аннотация из renames.tex
	


%% Не менять - Do not modify
\thispagestyle{empty}
%\setcounter{sumPageLast}{\value{page}} %сохранили номер последней страницы Задания
%\setcounter{sumPages}{\value{sumPageLast}-\value{sumPageFirst}}
%sumPageLast \arabic{sumPageLast}
%
%sumPages \arabic{sumPages}
%\restoregeometry % восстанавливаем настройки страницы
%\input{my_folder/summary_settings_restore}	% настройки - конец			 	 % Реферат 
										 % Убираем footnotes, дубли команд \abstractEn и \abstractRu 
										

\input{my_folder/contents}  	         % Оглавление


\chapter*{Введение} % * не проставляет номер
\addcontentsline{toc}{chapter}{Введение} % вносим в содержание

Мы повсеместно встречаемся с рекомендациями на различных сервисах. Это могут быть рекомендации фильмов, книг, музыки либо товаров. В вопросе рекомендаций остаются в выигрыше обе стороны взаимодействия – это компании предоставляющие свои товары или услуги, а также пользователи, которые пользуются товарами или услугами. От того насколько точны будут рекомендации, тем
быстрее и качественнее пользователи буду удовлетворять свои потребности, тем самым они сэкономят себе время, и будут оставаться лояльны компаниям предоставляющие свои услуги.


В основе работы рассматриваемой модели лежит гипотеза об дистрибутивности, которая заключается в том, что объекты, встречающие в схожих контекстах, имеют близкое значение \cite{harris1954distributional}.  Самой популярной моделью, основанной на данной гипотезе, является модель word2vec, позволяющая представлять слова в векторном пространстве. В данной работе объекты (объявления) представляются в многомерном векторном пространстве.


В данной работе используются структурированные объекты – это такие объекты реального мира, которые описываются конечным множеством признаков в виде таблицы объект – признак. Главной проблемой исследования таких данных является высокая степень разреженности матрицы объект – признак. В данной работе также будут рассмотрены способы борьбы с пропусками.


Главное преимущество рассматриваемого метода перед остальными методами заключается в том, что данный метод придает семантический смысл используемым объектам. Это позволяет более точно предсказывать наиболее похожие объекты между собой. 


Также особый интерес представляет устройство полученного векторного пространства, которое позволяет применять полностью весь математический аппарат для исследований и поиска закономерностей.


Стоит отметить также очень важный аспект данного исследования — это выявление семантических связей между объектами. Данное свойство имеет перспективы к дальнейшим исследованиям, направленных на выявление скрытых связей между объектами.

%% Вспомогательные команды - Additional commands
%\newpage % принудительное начало с новой страницы, использовать только в конце раздела
%\clearpage % осуществляется пакетом <<placeins>> в пределах секций
%\newpage\leavevmode\thispagestyle{empty}\newpage % 100 % начало новой строки	    	 % Введение

%% Начало основной части
\chapter{Постановка задачи} \label{ch1}
\newtheorem{definition}{Опредение}

% не рекомендуется использовать отдельную section <<введение>> после лета 2020 года
%\section{Введение. Сложносоставное название первого параграфа первой главы для~демонстрации переноса слов в содержании} \label{ch1:intro}
Перейдем к формальной постановке задаче.

В данном исследовании предоставлена выборка истории поведения обезличенных пользователей, а также информация об объектах, с которыми взаимодействовали пользователи. При каждом новом входе на веб-ресурс, пользователь начинает новую сессию, которая сохраняется в обезличенном варианте.

То есть для каждой пользовательской сессии известны идентификаторы объектов, для которых известны их признаковое описание.

Необходимо построить векторное представление объектов, которые помимо признакового описания хранили в себе еще и связь между историей взаимодействия пользователей.

Пусть $H$ - история поведения пользователей на веб-ресурсе за все время, $O$ - множество всех объектов присутствующих в базе данных веб-ресурса. $U$ - множество всех пользователей посещавших веб-ресурс.
Тогда поведения каждого пользователя ${u \in U}$ описывается следующим образом: ${({{{o_1}_h}^{h} \dots {{o_k}_h}^{h}})^{u}}$, где ${h \in H}$ - интервал времени, ${k_h}$ - длинна сессии пользователя за определенный интервал времени.


Задача построения векторного представления объектов заключается в сопоставление каждому объекту $o \in O$ вектора $\upsilon_o \in \mathbb{R}^{m}, m \ll |O| $. Такое отображение должно давать в результате такие вектора, чтобы похожие объекты были близки по расстоянию друг к другу.


Полученные векторные представления будут использоваться для:

\begin{itemize}
	\item Анализа пользовательских сессий;
	\item Кластеризации пользователей исходя из их поведения на веб-ресурсе;
	\item Построение рекомендательной системы;
\end{itemize}

\begin{definition}
	Рекомендательная система - это подкласс систем фильтрации информации, которая стремится предсказать «рейтинг» или «предпочтение», которое пользователь дал бы объекту \cite{ricci2011introduction}.
\end{definition}

Рекомендательная система представляет из себя задачу ранжирования.
Определим формально понятие задачи ранжирования:

\[ X - \text{множество объектов} \]

Имеется выборка, состоящая из n элементов:


\begin{equation}
	X^{n} = {x_1,\dots, x_n}
\end{equation}


Данные объекты содержат в себе признаковое описание.

В задаче ранжирования целевой переменной является пара вида:

\begin{equation}
	(i, j): x_i < x_j
\end{equation}

Необходимо построить ранжирующее отображение:

\begin{equation}
	f: X \rightarrow \mathbb{R} \text{ такую, что } i < j \Rightarrow f(x_i) < f(x_j)
\end{equation}

\section{Цель работы}
Целью данной работы является создание рекомендательной системы в основу которого ляжет векторное представление объектов.


\section{Методы оценки качества полученных векторов}
Суть векторного представления объектов в том, чтобы объекты находящиеся в одной сессии пользователя были близки по расстоянию друг к другу.
Для проверки этого свойства можно воспользоваться методами из задачи близости \cite{rubenstein1965contextual}. Но каждая задача близости привязана к конкретной выборке данных, поэтому данные методы не подходят для нашего исследования, так как исследуемые данные не являются публичными.


В статье \cite{mikolov2013efficient} описывается метод оценки полученных векторов путем проведения алгебраических операций, то есть поиска аналогий для векторов.
Пример:

\begin{equation}
	\upsilon_\text{царь} - \upsilon_\text{мальчик} + \upsilon_\text{девочка} \approx \upsilon_\text{царица}
\end{equation}

Известно, что для каждых 4 слов, первое находится в таком же семантическом отношении, как и третье с четвёртым (пример отношения: ягода - растение).
Но даже в этом случае необходимо, чтобы ассесоры разметили данные и тогда можно было бы оценить качество. По этой причине придется оценивать релевантность рекомендаций путем перекрестного сравнения с уже имеющийся системой рекомендаций на веб-ресурсе.


%% Вспомогательные команды - Additional commands
%
%\newpage % принудительное начало с новой страницы, использовать только в конце раздела
%\clearpage % осуществляется пакетом <<placeins>> в пределах секций
%\newpage\leavevmode\thispagestyle{empty}\newpage % 100 % начало новой страницы	         	 % Глава 1
\ContinueChapterBegin % размещать главы <<подряд>> 
\chapter{Обзор литературы} \label{ch2}
	
% не рекомендуется использовать отдельную section <<введение>> после лета 2020 года
%\section{Введение} \label{ch2:intro}

Исследуемая модель является Content-Based моделью рекомендаций, так как полученные вектора помимо своих характеристик хранят в себе и информацию о семантическом отношении в сессии пользователя.
Поэтому в данном параграфе будут рассматриваться популярные модели получения векторов и методы рекомендации основанные на модели.

\section{Модели получения векторного представления слова}
\subsection{Word2Vec (SGNS)} 

В основе работы данного алгоритма лежит идея о моделирование условного распределения вероятностей соседних слов. Также стоит отметить, что в отличие от других моделей дистрибутивной семантики (GloVe), Word2Vec работает с последовательностью слов, находящиеся от центрального слова на заданном расстояние - ширина окна.\\
В рассматриваемой модели хранятся и настраиваются два вектора для каждого слова. Первый вектор - является центральным представлением слова в рассматриваемом окне. Второй вектор - является контекстным представлением слова. \\
Для поиска оптимума в пространстве параметров данной модели используется градиентный спуск. \\

Skip Gram - предсказываем соседние слова по центральному слову\cite{word2vec}:

\begin{equation}
	W, D \in \mathbb{R}^{Vocab \times EmbSize} \\
	\sum_{CenterW_i} P(CtxW_{-2}, CtxW_{-1}, CtxW_{+1}, CtxW_{+2} | CenterW_i; W, D) \rightarrow \max_{W,D}
\end{equation}

Стоит отметить, что сумма в вышеописанном выражение идет не по всем уникальным словам, а по всем возможным словоупотреблениям.
    
Мы предполагаем, что соседние слова условно независимы друг от друга, когда мы уже рассмотрели центральное слово. 

\begin{equation}
	P(CtxW_{-2}, CtxW_{-1}, CtxW_{+1}, CtxW_{+2} | CenterW_i; W, D) = \prod_j P(CtxW_j | CenterW_i; W, D)
\end{equation}

Тогда наше распределение можно представить в виде произведения более простых распределений.

Каждое такое более простое распределение, будем моделировать при помощи softmax.


\begin{equation}
	P(CtxW_j | CenterW_i; W, D) = \frac{e^{w_i \cdot d_j}} { \sum_{j=1}^{|V|} e^{w_i \cdot d_j}} = softmax
\end{equation}


Из-за наличия в знаменателе суммы по всем объектам нашей выборки, каждый шаг градиентного спуска обходится вычислительно затратно. 
     
Поэтому будем использовать аппроксимацию negative sampling (отрицательное сэмплирование)

\begin{equation}
	P(CtxW_j | CenterW_i; W, D) \simeq \frac{e^{w_i \cdot d_j^+}} { \sum_{j=1}^{k} e^{w_i \cdot d_j^-}}, \quad k \ll |V|
\end{equation}


Суть данной аппроксимации заключается в том, что мы будем считать скалярное произведение в знаменателе не по всей нашей выборки объектов, а лишь по некоторым случайно выбранным.

\subsection{TF-IDF}

Данный подкласс моделей еще называется "мешком слов". Главная идея таких алгоритмов это то, что тематика текста хорошо описывается не порядком слов в документе, а
составом лексикона и частотой встречаемости слов.

Тогда каждый документ описывается разреженным вектором.
Для того, чтобы модель адекватно описывала данные необходимо, чтобы у каждого слова был свой вес. Одним из методов подсчета веса слова и является метод TF-IDF.

Основная идея в том, что чем чаще слово встречается в документе, тем более характерно оно для этого документа.
С другой стороны чем реже встречается слово в выборке документов, тем оно более специфично и информативно.

TF - term frequency - значимость слова в рамках документа \cite{tfidf}

\begin{equation}
	TF(w, d) = \frac{WordCount(w, d)}{Length(d)} 
\end{equation}

где $WordCount(w, d)$ - количество употреблений слова w в документе d,
$Length(d)$ - длина документа d в словах. 


IDF - inverse document frequency - специфичность слова \cite{tfidf}

\begin{equation}
	IDF(w, c) = \frac{Size(c)}{DocCount(w, c)}
\end{equation}

где $DocCount(w, c)$ - количество документов в коллекции c, в которых встречается слово w,

$Size(c)$ - размер коллекции c в документах.

Тогда вес слова подсчитывается следующим образом:


\begin{equation}
	TFIDF(w, d, c) = TF(w, d) * IDF(w, c)
\end{equation}

\section{Модель рекомендаций}
Главная цель моделей рекомендаций - это моделирование отношения между поведением пользователя и товарами или услугами предоставляемыми сервисами.

Отношение между пользователем и объектами можно приблизить некоторыми числами, которые описывают параметры пользователя и параметры объектов. Таким образом образуются векторы в пространстве одной и той же размерности,
при этом потребовав, чтобы скалярное произведение вектора, описывающего пользователя, и вектора, описывающий объект, хорошо приближала оценку отношений.
\begin{equation}
	x_{ij} \approx \langle u_i, \nu_j \rangle
\end{equation}
$u_i$ - параметры пользователя

$\nu_j$ - параметры объектов

Таким образом мы перешли к оптимизационной задаче:

\begin{equation}
	\sum (\langle u_i, \nu_j \rangle - x_{ij})^{2} \rightarrow min
\end{equation}

\subsection{SVD (Singular Value Decomposition)}

Пусть дана матрица пользователи - объекты, на пересечении которых стоят оценки пользователей.

Данная матрица имеет огромный размер (количество пользователей интернет ресурса может достигать нескольких миллионов, как и количество объектов предоставляемых веб-ресурсом)

Для любой вещественной $(n \times n)$ – матрицы $А$ существуют две вещественные ортогональные матрицы $U$ и $V$ такие, что 
\begin{equation}
	U^{T}AV = \varLambda 
\end{equation} \cite{svd}

Используя SVD-разложение матрицы пользователи - объекты, мы получим 2 матрицы: $U$ $n \times  k$  и $V$ $m \times k$, где
n - число пользователей, m -  число объектов, k - набор факторов.
Данные факторы и являются характеристикой вкусов и предпочтений пользователей.




%% Вспомогательные команды - Additional commands
%
%\newpage % принудительное начало с новой страницы, использовать только в конце раздела
%\clearpage % осуществляется пакетом <<placeins>> в пределах секций
%\newpage\leavevmode\thispagestyle{empty}\newpage % 100 % начало новой страницы	         	 % Глава 2
\newpage
\chapter{Предообработка и анализ данных} \label{ch3}

% не рекомендуется использовать отдельную section <<введение>> после лета 2020 года
%\section{Введение} \label{ch3:intro}

В данной главе будет описана обработка реальных данных, и методы борьбы с пропусками.
	
\section{Исходные данные} \label{ch3:sec1}

Для проведения экспериментов были предоставлены обезличенные данные о поведение пользователей на интернет-ресурсе «Яндекс.Недвижимость». Также была представлена вся информация об имеющийся на определённый момент времени объявлениях недвижимости.   \\
За представленный период 1.3 миллиона уникальных пользователей совершили 17 миллионов «кликов» (Клик – это взаимодействие пользователя с объектом интернет-ресурса, где объекты представляют из себя объявления о продаже/аренде недвижимости) по 0.4 миллионам объектам.

\section{Анализ данных} \label{ch3:sec2}

Для исследования были предоставлены табулированные данные.
\begin{enumerate}
    \item Данная таблица несет в себе информацию об истории активности пользователей на интернет-ресурсе, в частности их клики на объявления. Эта информация содержала в себе, уникальный идентификатор пользователя, уникальный идентификатор объявления, с которым взаимодействовал пользователь, а также временная отметка данного действия.
    
    \begin{figure}[h!]
        \centering
        \includegraphics[scale=0.5]
        {my_folder/images/offer_history.png}
        \caption{Структура таблицы истории пользователей}
        \label{fig:history_table}
    \end{figure}
    
    \item Также была предоставлена таблица в виде матрицы объект-признак. В ней перечисленны все характеристики объявлений за определенный промежуток времени. Она содержит в себе 380036 уникальных объявлений, описываемых 36 признаками
    
    \begin{figure}[h!]
        \centering
        \includegraphics[width=\textwidth]
        {my_folder/images/offer_details.png}
        \caption{Информация об объявлениях}
        \label{fig:details_table}
    \end{figure}

\end{enumerate}
В практических задачах, реальные данные не очень хорошие, так как они сильно разреженны. С этой проблемой необходимо бороться. \\
    \begin{figure}[h!]
        \centering
        \includegraphics[width=\textwidth]
        {my_folder/images/nulls.png}
        \caption{Процентное соотношения пропусков}
        \label{fig:my_label}
    \end{figure}
Обработка пропусков в данных - это отдельная обширная область в анализе данных. Существует множество методов, которые позволяют бороться с пропусками в данных, одним из них является построение решающих функций, которые будут предсказывать пропущенные значения. Но данное решение весьма трудоемко и не вкладывается в общий ход решения поставленной задачи. Поэтому приходилось каждую характеристику объявления обрабатывать отдельно. \\
Так например если не указана стоимость комиссия агента по продажам, то мы заполняли данный пропуск, как отсутствие комиссии. Также характеристику о том, что данное объявление выставляется впервые, в случае пропуска заполнялось как истина. \\
Остальные пропуски учитывались как самостоятельный элемент, эта эвристика выходит из принципа максимального правдоподобия. 
Также одной из важных подзадач в анализе данных было определение оптимальной максимальной длинны пользовательской сессии. Этот фактор имел несколько предпосылок: 
\begin{enumerate}
    \item Работа алгоритма word2vec на длинных сессиях было бы вычислительна затратная
    \item Длинные сессии могли быть сгенерированы ботами, которые обрабатывают интернет-ресурс, их активность не имеет интереса для нашей задачи.
\end{enumerate}

\begin{figure}[h!]
    \centering
    \includegraphics[width=\textwidth]
    {my_folder/images/distribution of clicks in session.png}
    \caption{Распределение среднего числа кликов пользователей в сессии}
    \label{fig:my_label}
\end{figure}

\begin{figure}[h!]
    \centering
    \includegraphics[width=\textwidth]
    {my_folder/images/average_time_session.png}
    \caption{Распределение средней разности времени между кликами пользователей в сессии}
    \label{fig:my_label}
\end{figure}


\FloatBarrier % заставить рисунки и другие подвижные (float) элементы остановиться

%% Вспомогательные команды - Additional commands
%
%\newpage % принудительное начало с новой страницы, использовать только в конце раздела
%\clearpage % осуществляется пакетом <<placeins>> в пределах секций
%\newpage\leavevmode\thispagestyle{empty}\newpage % 100 % начало новой страницы           	 % Глава 3
\newpage
\chapter{Модели исследования и результаты} \label{ch4}

% не рекомендуется использовать отдельную section <<введение>> после лета 2020 года
%\section{Введение} \label{ch4:intro}

В данной главе будут рассмотрены  исследуемые модели машинного обучения. Результаты их работы и сравнение рекомендаций полученных в результате работы рассмотренных моделей.

	
\section{Предообучение признаков объекта} \label{ch4:sec1}

В данной модели моделируется условное распределение вероятностей признаков, находящихся в соседних объектах пользовательских сессий. 


Признаки, имеющие шкалу абсолютных величин, разбивались на перцентили, данная эвристика исходила из того, что в задачах рекомендации пользователи в общей совокупности делятся на некоторые подмножества, 
которые обуславливаются общими характеристиками по некоторым признакам. То есть множества сущностей всех признаков мы разбивали на категории.


Каждый признак имел векторное представление размером $min(n / 2 + 1, 50)$, $n$ - мощность множества сущностей признака объекта.
Данная эмпирическая закономерность была выведена путем поиска по сетке, описанная в \cite{sizeEmbedding}.


После чего полученные вектора сущностей признаков конкатенировались в соответствие с матрицей объект-признак. 
Тем самым мы получем векторное представление для каждого объекта.

\begin{figure}[!ht]
    \centering
    \includegraphics[width=\textwidth]
    {my_folder/images/tensors.png}
    \caption{Архитектура модели предобучения признаков}
    \label{fig:tensors}
\end{figure}


Полученные вектора признаков будут использоваться как новые признаки для следующих моделей. 

График функции потерь при обучении модели для 4 признаков: 

\section{Использование полносвязных слоев} \label{ch4:sec2}

Данная модель по своей структуре является классификатором.

Его идея заключается в том, чтобы найти взаимосвязь между предобученными признаками объекта и представить в более меньшей размерности.

\begin{figure}[!ht]
    \centering
    \includegraphics[width=\textwidth]
    {my_folder/images/neural_network.png}
    \caption{Архитектура нейронной сети}
    \label{fig:neural_network}
\end{figure}

\subsection{Метод обучения} \label{ch4:sec2:sec1}

Для обучения модели необходимо выбрать пару из нашей выборки, которая входит в сессию пользователя. 
К данной паре необходимо подобрать некоторое количество негативных примеров. Это случайные объекты из нашей выборки.

Затем прогоняем через сеть пару объектов, а также негативные примеры. После чего считаем скалярное произведение между парой и негативными элементами.
Выбираем 100 элементов с самым большим значением скалярного произведения из множества негативных примеров. И считаем функцию потерь - cross entropy loss.

\begin{equation}
    loss(x, class) = -\log (\frac{\exp (x[class])}{\sum (\exp (x[j]))} ) = -x[class] + \log (\sum(exp(x[j])))
\end{equation}

\begin{figure}[!ht]
    \centering
    \includegraphics[width=\textwidth]
    {my_folder/images/train.png}
    \caption{Структура тренировки модели}
    \label{fig:train}
\end{figure}

\section{Модифицированная модель с полносвязными слоями} \label{ch4:sec3}

В данной модели усовершенствование работы алгоритма заключается из того, что для полученных вектора уже хранят в себе семантический смысл, 
поэтому для каждой сессии пользователя, используются алгебраические операции над векторами объектов. То есть теперь данными являются не пара похожих объектов, а пара: среднее векторов между последовательностью объектов входящих в одну сессию и следующим за ними объектом из этой же сессии. 

\section{Ансамблирование алгоритмов обучения} \label{ch4:sec4}

В данной модели моделируется условное распределение вероятностей объектов, находящихся в соседних объектах пользовательских сессий.
При этом имея в наличии уже предобученные вектора признаков.

Данное распределение моделировалось на таком же алгоритме word2vec, как и для обучения признаков объектов.
Отличие заключается лишь в том, что изначальные вектора были получены в результате конкатенации векторов-признаков, а не инициализации случайным шумом.

\section{Выбор метрики} \label{ch4:sec5}
В результате применение вышеописанного алгоритма, мы сможем получить векторное представление объектов более низкой размерности,
 при этом данный алгоритм смог сохранить семантику объектов в векторном представление.
  Эта семантика выражается через отношение близости объектов. Наиболее схожие объекты между собой находятся близко в построенном векторном пространстве.
   Эта идея является ключевой для построения рекомендательной системы. \\
В статье  \cite{dist} описываются сравниваются различные оценки близости такие как: евклидово расстояние, косинусная близость, метрика Манхэттена, расстояние Бхаттачарья, расстояние Хеллингера, дивергенция Кульбака-Лейблера. В результате экспериментов косинусная близость показала наилучший результат. \\
 
\[
\displaystyle {\text{similarity}}=\cos(\theta )={\mathbf {A} \cdot \mathbf {B}  \over \|\mathbf {A} \|\|\mathbf {B} \|}={\frac {\sum \limits _{i=1}^{n}{A_{i}B_{i}}}{{\sqrt {\sum \limits _{i=1}^{n}{A_{i}^{2}}}}{\sqrt {\sum \limits _{i=1}^{n}{B_{i}^{2}}}}}}
\]

\section{Результаты моделей} \label{ch4:sec6}

\subsection{Графики функции потерь} \label{ch4:sec6:sec1}

\begin{enumerate}
    \item Векторное представление признаков объекта.
    \begin{figure}[!ht]
        \centering
        \includegraphics[scale=0.5]
        {my_folder/images/lotarea_loss.png}
        \caption{Функция потерь для площади земельного участка}
        \label{fig:lotarea_loss}
    \end{figure}

    \begin{figure}[!ht]
        \centering
        \includegraphics[scale=0.5]
        {my_folder/images/live_value.png}
        \caption{Функция потерь для жилплощади}
        \label{fig:live_value}
    \end{figure}

    \begin{figure}[!ht]
        \centering
        \includegraphics[scale=0.5]
        {my_folder/images/kitchen_loss.png}
        \caption{Функция потерь для площади кухни}
        \label{fig:kitchen_space}
    \end{figure}

    \begin{figure}[!ht]
        \centering
        \includegraphics[scale=0.5]
        {my_folder/images/year.png}
        \caption{Функция потерь для года постройки}
        \label{fig:year}
    \end{figure}

    Можно увидеть, что обучение на всех признаков достигает некоторого оптимума.

    \item Модель с полносвязными слоями

    \begin{figure}[!ht]
        \centering
        \includegraphics[scale=0.8]
        {my_folder/images/neural.png}
        \caption{Функция потерь модели с полносвязными слоями}
        \label{fig:neural}
    \end{figure}

    \item Модифицированная модель с полносвязными слоями
    
    \begin{figure}[!ht]
        \centering
        \includegraphics[scale=0.8]
        {my_folder/images/update_neural.png}
        \caption{Функция потерь модели с полносвязными слоями}
        \label{fig:update_neural}
    \end{figure}

    \item Ансамблирование алгоритмов обучения
    

    \begin{figure}[!ht]
        \centering
        \includegraphics[scale=0.8]
        {my_folder/images/stacking.png}
        \caption{Функция потерь для ансамбля моделей}
        \label{fig:stacking}
    \end{figure}

    Как можно заметить модель быстро переобучается и теряет "знания" о предобученных признаках.

\end{enumerate}

\subsection{Поиск похожих объектов}

В данном исследовании будет рассмотрено, как алгоритм хорошо представляет объекты в векторном пространстве, то есть
располагает похожие объекты близко. Для этого будем строить рекомендации для некоторых случайно выбранных объектов, которые относится к разным категориям, таким как:

\begin{itemize}
    \item Продажа квартиры
    \item Аренда квартиры
    \item Аренда дома
\end{itemize} 


        \begin{figure}[ht!]
            \centering
            \includegraphics[width=\textwidth]
            {my_folder/images/tensor_pokupka.png}
            \caption{Рекомендация векторного представления признаков для продажи квартиры}
            \label{fig:tensor_pokupka}
        \end{figure}

        \newpage

        \begin{figure}[ht!]
            \centering
            \includegraphics[width=\textwidth]
            {my_folder/images/tensor_arenda.png}
            \caption{Рекомендация векторного представления признаков для аренды квартиры}
            \label{fig:tensor_arenda}
        \end{figure}
        \begin{figure}[ht!]
            \centering
            \includegraphics[width=\textwidth]
            {my_folder/images/tensor_house.png}
            \caption{Рекомендация векторного представления признаков для аренды загородного дома}
            \label{fig:tensor_house}
        \end{figure}

        \begin{figure}[ht!]
            \centering
            \includegraphics[width=\textwidth]
            {my_folder/images/neural_pokupka.png}
            \caption{Рекомендация модели с полносвязными слоями для продажи квартиры}
            \label{fig:neural_pokupka}
        \end{figure}
        \begin{figure}[ht!]
            \centering
            \includegraphics[width=\textwidth]
            {my_folder/images/neural_arenda.png}
            \caption{Рекомендация модели с полносвязными слоями для аренды квартиры}
            \label{fig:neural_arenda}
        \end{figure}
        \begin{figure}[ht!]
            \centering
            \includegraphics[width=\textwidth]
            {my_folder/images/neural_house.png}
            \caption{Рекомендация модели с полносвязными слоями для аренды загородного дома}
            \label{fig:neural_house}
        \end{figure}

    \begin{figure}[ht!]
        \centering
        \includegraphics[width=\textwidth]
        {my_folder/images/mod_pokupka.png}
        \caption{Рекомендация модифицированной модели с полносвязными слоями для продажи квартиры}
        \label{fig:mod_pokupka}
    \end{figure}
    \begin{figure}[ht!]
        \centering
        \includegraphics[width=\textwidth]
        {my_folder/images/mod_arenda.png}
        \caption{Рекомендация модифицированной модели с полносвязными слоями для аренды квартиры}
        \label{fig:mod_arenda}
    \end{figure}
    \begin{figure}[ht!]
        \centering
        \includegraphics[width=\textwidth]
        {my_folder/images/mod_house.png}
        \caption{Рекомендация модифицированной модели с полносвязными слоями для аренды загородного дома}
        \label{fig:mod_house}
    \end{figure}

    \begin{figure}[ht!]
        \centering
        \includegraphics[width=\textwidth]
        {my_folder/images/2vec_pokupka.png}
        \caption{Рекомендация ансамбля моделей для продажи квартиры}
        \label{fig:2vec_pokupka}
    \end{figure}
    \begin{figure}[ht!]
        \centering
        \includegraphics[width=\textwidth]
        {my_folder/images/2vec_arenda.png}
        \caption{Рекомендация ансамбля моделей для аренды квартиры}
        \label{fig:2vec_arenda}
    \end{figure}
    \begin{figure}[ht!]
        \centering
        \includegraphics[width=\textwidth]
        {my_folder/images/2vec_house.png}
        \caption{Рекомендация ансамбля моделей для аренды загородного дома}
        \label{fig:2vec_house}
    \end{figure}

% Модели с полносвязными слоями показались адекватны данным, также можно заметить и семантическую связь между элементами рекомендацией, это исходит из того,
% что модель выдает рекомендации основанные на характеристиках объектов и в последнюю очередь опирается на расстояние между интересующим объектом.

% Модель состоящая только из предобученных признаков вызывает сомнения, так как результатом поиска ближайших соседей является просто подбором близких по характеристикам объектам.  

% Модель использующая ансамбль - показала наихудший результат. Рекомендации - никак невозможно интерпретировать. 

\newpage
\newpage
\subsection{Кластеризация}

В данной секции экспериментов было исследовано выделение кластеров из построенного векторного пространства.
Было выбрано первые 1000 пользовательских взаимодействий (объектов), и на векторном представлении данных объектов выполнялся алгоритм кластеризации - K-Means\cite{likas2003global}
Так как алгоритм кластеризации запускался с фиксированными параметрами, но с разными векторными пространствами, то далее для интерпретации будет использоваться кластер под номером 5. 

Чтобы оценивать качество кластеризации при одних и тех же параметров, я использовал Силуэт (англ. Silhouette) \cite{sil}

\begin{equation}
    Sil(С) = \dfrac{1}{N} \sum_{c_k \in C} \sum_{x_i \in c_k} \dfrac{ b(x_i, c_k) - a(x_i, c_k) }{ max \{ a(x_i, c_k), b(x_i, c_k) \}  }
\end{equation}

Чем ближе данная оценка к 1, тем лучше.

\paragraph{Предобученные признаки}
Элементы кластера являются интерпретируемыми. Можно сказать, что в данный кластер попали те объекты, которые находятся в исторических районах города и имеют ремонт жилища.

$Sil(C) = 0.065$

\begin{figure}[ht!]
    \centering
    \includegraphics[scale=0.7]
    {my_folder/images/tensor_rock.png}
    \caption{График каменистой осыпи для определения числа кластеров}
    \label{fig:tensor_rock}
\end{figure}

\begin{figure}[ht!]
    \centering
    \includegraphics[scale=0.7]
    {my_folder/images/tensor_img.png}
    \caption{Представление кластеризации 1000 объектов при помощи PCA\cite{pca}}
    \label{fig:tensor_img}
\end{figure}

\begin{figure}[ht!]
    \centering
    \includegraphics[width=\textwidth]
    {my_folder/images/tensor_cluster.png}
    \caption{Элементы кластера}
    \label{fig:tensor_cluster}
\end{figure}

\newpage
\paragraph{Модель с полносвязными слоями}
Представление пространства при помощи PCA при использование данной модели, дает более равномерное покрытие пространства точками.
Интерпретация кластера в данном случае является не столь тривиальной, но логика все равно прослеживается. Можно сказать, что в данном кластере сосредоточены аренда квартир в определенной ценовой категории. 

$Sil(C) = 0.053$

\begin{figure}[ht!]
    \centering
    \includegraphics[scale=0.7]
    {my_folder/images/neural_rock.png}
    \caption{График каменистой осыпи для определения числа кластеров}
    \label{fig:neural_rock}
\end{figure}

\begin{figure}[ht!]
    \centering
    \includegraphics[scale=0.7]
    {my_folder/images/neural_img.png}
    \caption{Представление кластеризации 1000 объектов при помощи PCA\cite{pca}}
    \label{fig:neural_img}
\end{figure}

\begin{figure}[ht!]
    \centering
    \includegraphics[width=\textwidth]
    {my_folder/images/neural_cluster.png}
    \caption{Элементы кластера}
    \label{fig:neural_cluster}
\end{figure}

\newpage
\paragraph{Модифицировання модель с полносвязными слоями}
В данном случае, полученное векторное представление не такое равномерное как у обычной модели с полносвязными слоями, оно похоже на пространство векторов предобученных признаков объекта.
Кластер также вызывает трудности при интерпретации, но общее сходство у объектов найти можно.

$Sil(C) = 0.056$

\begin{figure}[ht!]
    \centering
    \includegraphics[scale=0.7]
    {my_folder/images/mod_rock.png}
    \caption{График каменистой осыпи для определения числа кластеров}
    \label{fig:mod_rock}
\end{figure}

\begin{figure}[ht!]
    \centering
    \includegraphics[scale=0.7]
    {my_folder/images/mod_img.png}
    \caption{Представление кластеризации 1000 объектов при помощи PCA\cite{pca}}
    \label{fig:mod_img}
\end{figure}

\begin{figure}[ht!]
    \centering
    \includegraphics[width=\textwidth]
    {my_folder/images/mod_cluster.png}
    \caption{Элементы кластера}
    \label{fig:mod_cluster}
\end{figure}
\FloatBarrier % заставить рисунки и другие подвижные (float) элементы остановиться

\FloatBlock
\section{Выводы} \label{ch4:conclusion}

Исходя из проделанных экспериментов можно сделать вывод о том, что рассмотренные модели, кроме модели, использующая ансамбль, адекватны данным.
Косинусная мера в построенных векторных пространствах дает интерпретируемые рекомендации. 

Модель с полносвязными слоями исходя из перекрестной проверки дает лучшие рекомендации, а также более равномерно заполняет векторное пространство объектами.
Модификация этой модели не привнесла никаких изменений, а лишь усложнила интерпретацию алгоритмов кластеризации. 

Модель основанная на предобученных признаках также дает, адекватные рекомендации объектов. 

Модель, основанная на ансамбле моделей, показала наихудший результат. Исходя из графика функции потерь, было видно, как модель быстро переобучается и теряет связь с изначальными характеристиками объектов, и выучивает лишь их ранги в пользовательской сессии. 


%% Вспомогательные команды - Additional commands
%
%\newpage % принудительное начало с новой страницы, использовать только в конце раздела
%\clearpage % осуществляется пакетом <<placeins>> в пределах секций
%\newpage\leavevmode\thispagestyle{empty}\newpage % 100 % начало новой страницы           	 % Глава 3
\ContinueChapterEnd % завершить размещение глав <<подряд>>
%% Завершение основной части

\chapter*{Заключение} \label{ch-conclusion}
\addcontentsline{toc}{chapter}{Заключение}	% в оглавление 

В результате выполнения выпускной квалификационной работы удалось рассмотреть актуальные методы векторизации слов, 
применить эти методы к структурированным объектам, и исследовать модели на реальной выборке данных.

В ходе исследования моделей векторизации слов было выяснено, что получение векторного представление любого структурированного объекта - осуществима, 
необходимо лишь иметь данные об историческом взаимодействии с рассматриваемыми объектами.

Построенная модель на основе полносвязных слоев дает релевантные рекомендации, оцененные при помощи перекрестной проверки.
Также предобученные признаки объектов можно использовать на вход более мощным моделям построения рекомендаций.

Программная реализация была выполнена на языке Python, она состояла из pipeline обработки данных, обучения модели и построения рекомендаций для изначально выбранных объектов.


Проведенную работу можно считать успешной, так как выполнена основная задача работы - это построение рекомендательной системы,
основанной на векторном представлении структурированных объектов. 


Дальнейшие исследования заключается в проведение A/B тестирования на реальных пользователях веб-ресурса и изучение поведения модели при большем объеме данных.        	 % Заключение

%% Наличие следующих перечней не исключает расшифровку сокращения и условного обозначения при первом упоминании в тексте!
% \input{my_folder/acronyms}		         % Необязательная рубрика! Список сокращений и условных обозначений

% \input{my_folder/dictionary}    		 % Необязательная рубрика! Словарь терминов
% По порядку после Списка сокращений и условных обозначений, если есть.	


%%% Не мянять - Do not modify
%%
%%
\clearpage                                  % В том числе гарантирует, что список литературы в оглавлении будет с правильным номером страницы
% \hypersetup{ urlcolor=black }               % Ссылки делаем чёрными
%\providecommand*{\BibDash}{}                % В стилях ugost2008 отключаем использование тире как разделителя 
\urlstyle{rm}                               % ссылки URL обычным шрифтом
\ifdefmacro{\microtypesetup}{\microtypesetup{protrusion=false}}{} % не рекомендуется применять пакет микротипографики к автоматически генерируемому списку литературы
%\newcommand{\fullbibtitle}{Список литературы} % (ГОСТ Р 7.0.11-2011, 4)
%\insertbibliofull  
%\noindent
%\begin{group}
\chapter*{Список использованных источников}	
\label{references}
\addcontentsline{toc}{chapter}{Список использованных источников}	% в оглавление 
\printbibliography[env=SSTfirst]                         % Подключаем Bib-базы
%\ifdefmacro{\microtypesetup}{\microtypesetup{protrusion=true}}{}
%\urlstyle{tt}                               % возвращаем установки шрифта ссылок URL
%\hypersetup{ urlcolor={urlcolor} }          % Восстанавливаем цвет ссылок



%\urlstyle{rm}                               % ссылки URL обычным шрифтом
%\ifdefmacro{\microtypesetup}{\microtypesetup{protrusion=false}}{} % не рекомендуется применять пакет микротипографики к автоматически генерируемому списку литературы
%\insertbibliofull                           % Подключаем Bib-базы
%\ifdefmacro{\microtypesetup}{\microtypesetup{protrusion=true}}{}
%\urlstyle{tt}                               % возвращаем установки шрифта ссылок URL
		     % Список литературы

% Здесь можно поместить список иллюстративного материала

\appendix % не редактировать / keep unmodified


% \input{my_folder/appendix1}			     % Приложение 1

% \input{my_folder/appendix2}			 	 % Приложение 2


\end{document} % конец документа


%%% Удачной защиты ВКР! - Good luck on the thesis defense!
%%
%%% Поддержать проект
%%
%% Запросы на добавление / изменение просим писать на следующей странице:
%% https://github.com/ParkhomenkoV/SPbPU-student-thesis-template/issues
%%
%% Список пожеланий в файле шаблона <<TO-DO-list.tex>>
%%
%% Благодарности просим указывать в виде 
%%
%% 1. Добавление <<Звезды>> проекту https://github.com/ParkhomenkoV/SPbPU-student-thesis-template/stargazers
%%
%% 2. Добавления <<Сердечка>> и репоста проекта в социальных сетях:
%%		https://vk.com/latex_polytech 
%%		https://www.fb.com/groups/latex.polytech
%%

%%% Support project
%%
%% Requests on adding / modifications is better to be publishen on the following web-page:
%% https://github.com/ParkhomenkoV/SPbPU-student-thesis-template/issues
%%
%% Wishlist is in the template's file called <<TO-DO-list.tex>>
%%
%% Acknowledgements are better to be done in the form of 
%%
%% 1. Adding <<Star>> to the project https://github.com/ParkhomenkoV/SPbPU-student-thesis-template/stargazers
%%
%% 2. Adding <<Likes>> and Project repost in the social networks:
%%		https://vk.com/latex_polytech 
%%		https://www.fb.com/groups/latex.polytech
%% 

% Check list при передаче ВКР:
% - Количество страниц в Задании 2. Если нет, то комментирование последней строки в my_task.tex
% - Зачистка всех вспомогательных файлов (Clear auxilary files) и компиляция ВКР не менее 3х раз