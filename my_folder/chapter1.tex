\chapter{Постановка задачи} \label{ch1}
\newtheorem{definition}{Опредение}

% не рекомендуется использовать отдельную section <<введение>> после лета 2020 года
%\section{Введение. Сложносоставное название первого параграфа первой главы для~демонстрации переноса слов в содержании} \label{ch1:intro}
Перейдем к формальной постановке задаче.

В данном исследовании предоставлена выборка истории поведения обезличенных пользователей, а также информация об объектах, с которыми взаимодействовали пользователи. При каждом новом входе на веб-ресурс, пользователь начинает новую сессию, которая сохраняется в обезличенном варианте.

То есть для каждой пользовательской сессии известны идентификаторы объектов, для которых известны их признаковое описание.

Необходимо построить векторное представление объектов, которые помимо признакового описания хранили в себе еще и связь между историей взаимодействия пользователей.

Пусть $H$ - история поведения пользователей на веб-ресурсе за все время, $O$ - множество всех объектов присутствующих в базе данных веб-ресурса. $U$ - множество всех пользователей посещавших веб-ресурс.
Тогда поведения каждого пользователя ${u \in U}$ описывается следующим образом: ${({{{o_1}_h}^{h} \dots {{o_k}_h}^{h}})^{u}}$, где ${h \in H}$ - интервал времени, ${k_h}$ - длинна сессии пользователя за определенный интервал времени.


Задача построения векторного представления объектов заключается в сопоставление каждому объекту $o \in O$ вектора $\upsilon_o \in \mathbb{R}^{m}, m \ll |O| $. Такое отображение должно давать в результате такие вектора, чтобы похожие объекты были близки по расстоянию друг к другу.


Полученные векторные представления будут использоваться для:

\begin{itemize}
	\item Анализа пользовательских сессий;
	\item Кластеризации пользователей исходя из их поведения на веб-ресурсе;
	\item Построение рекомендательной системы;
\end{itemize}

\begin{definition}
	Рекомендательная система - это подкласс систем фильтрации информации, которая стремится предсказать «рейтинг» или «предпочтение», которое пользователь дал бы объекту \cite{ricci2011introduction}.
\end{definition}

Рекомендательная система представляет из себя задачу ранжирования.
Определим формально понятие задачи ранжирования:

\[ X - \text{множество объектов} \]

Имеется выборка, состоящая из n элементов:


\begin{equation}
	X^{n} = {x_1,\dots, x_n}
\end{equation}


Данные объекты содержат в себе признаковое описание.

В задаче ранжирования целевой переменной является пара вида:

\begin{equation}
	(i, j): x_i < x_j
\end{equation}

Необходимо построить ранжирующее отображение:

\begin{equation}
	f: X \rightarrow \mathbb{R} \text{ такую, что } i < j \Rightarrow f(x_i) < f(x_j)
\end{equation}

\section{Цель работы}
Целью данной работы является создание рекомендательной системы в основу которого ляжет векторное представление объектов.


\section{Методы оценки качества полученных векторов}
Суть векторного представления объектов в том, чтобы объекты находящиеся в одной сессии пользователя были близки по расстоянию друг к другу.
Для проверки этого свойства можно воспользоваться методами из задачи близости \cite{rubenstein1965contextual}. Но каждая задача близости привязана к конкретной выборке данных, поэтому данные методы не подходят для нашего исследования, так как исследуемые данные не являются публичными.


В статье \cite{mikolov2013efficient} описывается метод оценки полученных векторов путем проведения алгебраических операций, то есть поиска аналогий для векторов.
Пример:

\begin{equation}
	\upsilon_\text{царь} - \upsilon_\text{мальчик} + \upsilon_\text{девочка} \approx \upsilon_\text{царица}
\end{equation}

Известно, что для каждых 4 слов, первое находится в таком же семантическом отношении, как и третье с четвёртым (пример отношения: ягода - растение).
Но даже в этом случае необходимо, чтобы ассесоры разметили данные и тогда можно было бы оценить качество. По этой причине придется оценивать релевантность рекомендаций путем перекрестного сравнения с уже имеющийся системой рекомендаций на веб-ресурсе.


%% Вспомогательные команды - Additional commands
%
%\newpage % принудительное начало с новой страницы, использовать только в конце раздела
%\clearpage % осуществляется пакетом <<placeins>> в пределах секций
%\newpage\leavevmode\thispagestyle{empty}\newpage % 100 % начало новой страницы