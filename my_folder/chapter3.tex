\newpage
\chapter{Предообработка и анализ данных} \label{ch3}

% не рекомендуется использовать отдельную section <<введение>> после лета 2020 года
%\section{Введение} \label{ch3:intro}

В данной главе будет описана обработка реальных данных, и методы борьбы с пропусками.
	
\section{Исходные данные} \label{ch3:sec1}

Для проведения экспериментов были предоставлены обезличенные данные о поведение пользователей на интернет-ресурсе «Яндекс.Недвижимость». Также была представлена вся информация об имеющийся на определённый момент времени объявлениях недвижимости.   \\
За представленный период 1.3 миллиона уникальных пользователей совершили 17 миллионов «кликов» (Клик – это взаимодействие пользователя с объектом интернет-ресурса, где объекты представляют из себя объявления о продаже/аренде недвижимости) по 0.4 миллионам объектам.

\section{Анализ данных} \label{ch3:sec2}

Для исследования были предоставлены табулированные данные.
\begin{enumerate}
    \item Данная таблица несет в себе информацию об истории активности пользователей на интернет-ресурсе, в частности их клики на объявления. Эта информация содержала в себе, уникальный идентификатор пользователя, уникальный идентификатор объявления, с которым взаимодействовал пользователь, а также временная отметка данного действия.
    
    \begin{figure}[h!]
        \centering
        \includegraphics[scale=0.5]
        {my_folder/images/offer_history.png}
        \caption{Структура таблицы истории пользователей}
        \label{fig:history_table}
    \end{figure}
    
    \item Также была предоставлена таблица в виде матрицы объект-признак. В ней перечисленны все характеристики объявлений за определенный промежуток времени. Она содержит в себе 380036 уникальных объявлений, описываемых 36 признаками
    
    \begin{figure}[h!]
        \centering
        \includegraphics[width=\textwidth]
        {my_folder/images/offer_details.png}
        \caption{Информация об объявлениях}
        \label{fig:details_table}
    \end{figure}

\end{enumerate}
В практических задачах, реальные данные не очень хорошие, так как они сильно разреженны. С этой проблемой необходимо бороться. \\
    \begin{figure}[h!]
        \centering
        \includegraphics[width=\textwidth]
        {my_folder/images/nulls.png}
        \caption{Процентное соотношения пропусков}
        \label{fig:my_label}
    \end{figure}
Обработка пропусков в данных - это отдельная обширная область в анализе данных. Существует множество методов, которые позволяют бороться с пропусками в данных, одним из них является построение решающих функций, которые будут предсказывать пропущенные значения. Но данное решение весьма трудоемко и не вкладывается в общий ход решения поставленной задачи. Поэтому приходилось каждую характеристику объявления обрабатывать отдельно. \\
Так например если не указана стоимость комиссия агента по продажам, то мы заполняли данный пропуск, как отсутствие комиссии. Также характеристику о том, что данное объявление выставляется впервые, в случае пропуска заполнялось как истина. \\
Остальные пропуски учитывались как самостоятельный элемент, эта эвристика выходит из принципа максимального правдоподобия. 
Также одной из важных подзадач в анализе данных было определение оптимальной максимальной длинны пользовательской сессии. Этот фактор имел несколько предпосылок: 
\begin{enumerate}
    \item Работа алгоритма word2vec на длинных сессиях было бы вычислительна затратная
    \item Длинные сессии могли быть сгенерированы ботами, которые обрабатывают интернет-ресурс, их активность не имеет интереса для нашей задачи.
\end{enumerate}

\begin{figure}[h!]
    \centering
    \includegraphics[width=\textwidth]
    {my_folder/images/distribution of clicks in session.png}
    \caption{Распределение среднего числа кликов пользователей в сессии}
    \label{fig:my_label}
\end{figure}

\begin{figure}[h!]
    \centering
    \includegraphics[width=\textwidth]
    {my_folder/images/average_time_session.png}
    \caption{Распределение средней разности времени между кликами пользователей в сессии}
    \label{fig:my_label}
\end{figure}


\FloatBarrier % заставить рисунки и другие подвижные (float) элементы остановиться

%% Вспомогательные команды - Additional commands
%
%\newpage % принудительное начало с новой страницы, использовать только в конце раздела
%\clearpage % осуществляется пакетом <<placeins>> в пределах секций
%\newpage\leavevmode\thispagestyle{empty}\newpage % 100 % начало новой страницы