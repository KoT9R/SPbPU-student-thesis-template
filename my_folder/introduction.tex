\chapter*{Введение} % * не проставляет номер
\addcontentsline{toc}{chapter}{Введение} % вносим в содержание

Мы повсеместно встречаемся с рекомендациями на различных сервисах. Это могут быть рекомендации фильмов, книг, музыки либо товаров. В вопросе рекомендаций остаются в выигрыше обе стороны взаимодействия – это компании предоставляющие свои товары или услуги, а также пользователи, которые пользуются товарами или услугами. От того насколько точны будут рекомендации, тем
быстрее и качественнее пользователи буду удовлетворять свои потребности, тем самым они сэкономят себе время, и будут оставаться лояльны компаниям предоставляющие свои услуги.


В основе работы рассматриваемой модели лежит гипотеза об дистрибутивности, которая заключается в том, что объекты, встречающие в схожих контекстах, имеют близкое значение \cite{harris1954distributional}.  Самой популярной моделью, основанной на данной гипотезе, является модель word2vec, позволяющая представлять слова в векторном пространстве. В данной работе объекты (объявления) представляются в многомерном векторном пространстве.


В данной работе используются структурированные объекты – это такие объекты реального мира, которые описываются конечным множеством признаков в виде таблицы объект – признак. Главной проблемой исследования таких данных является высокая степень разреженности матрицы объект – признак. В данной работе также будут рассмотрены способы борьбы с пропусками.


Главное преимущество рассматриваемого метода перед остальными методами заключается в том, что данный метод придает семантический смысл используемым объектам. Это позволяет более точно предсказывать наиболее похожие объекты между собой. 


Также особый интерес представляет устройство полученного векторного пространства, которое позволяет применять полностью весь математический аппарат для исследований и поиска закономерностей.


Стоит отметить также очень важный аспект данного исследования — это выявление семантических связей между объектами. Данное свойство имеет перспективы к дальнейшим исследованиям, направленных на выявление скрытых связей между объектами.

%% Вспомогательные команды - Additional commands
%\newpage % принудительное начало с новой страницы, использовать только в конце раздела
%\clearpage % осуществляется пакетом <<placeins>> в пределах секций
%\newpage\leavevmode\thispagestyle{empty}\newpage % 100 % начало новой строки